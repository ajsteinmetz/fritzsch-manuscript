%%%%%%%%%%%%%%%%%%%%%%%%%%%%%%%%%%%%%%%%%%%%%%%%%%%%%%%%%%%%%%%%%%%%%%%%%%
%% Review Volume (last updated on 2014/03/05) %%
%% Trim Size: 9.61in x 6.69in %%
%% Text Area: 8in (include runningheads) x 5in %%
%% Main Text: 10 on 13pt %%
%% For support: Yolande Koh, <ykoh@wspc.com.sg> %%
%% D. Rajesh Babu, <rajesh@wspc.com.sg> %%
%%%%%%%%%%%%%%%%%%%%%%%%%%%%%%%%%%%%%%%%%%%%%%%%%%%%%%%%%%%%%%%%%%%%%%%%%%
%%
%\documentclass[wsdraft]{ws-rv961x669} % to draw border line around text area
%\documentclass{ws-rv961x669}
\documentclass[addchapnum]{ws-rv961x669} % to add chapter number in volume
\usepackage{ws-rv-van} % numbered citation/references (default)
%\usepackage{ws-rv-thm} % comment this line when `amsthm / theorem / ntheorem` package is used
%\usepackage{subfigure} % required only when side-by-side / subfigures are used
\usepackage{ws-index} % required only when multiple indexes are used
%\usepackage[colorlinks=false]{hyperref}
%\usepackage{doi}
\usepackage{bbm}
\usepackage{amsmath}
\usepackage{amssymb}
\makeindex
\newindex{aindx}{adx}{and}{Author Index} % author index
\renewindex{default}{idx}{ind}{Subject Index} % subject index

\newcommand{\req}[1]{Eq.~(\ref{#1})}
\newcommand{\rsec}[1]{Sec.~{\ref{#1}}}

\begin{document}

\chapter[Dynamic flavor mixing through transition moments]{\label{JR_ch1}Dynamic flavor mixing through transition moments}

\author[J. Rafelski, A. Steinmetz, and C. T. Yang]{Johann Rafelski, Andrew Steinmetz, and Cheng Tao Yang}
\aindx{Rafelski, J.}
\aindx{Steinmetz, A.}
\aindx{Yang, C. T.}

\address{Department of Physics, The University of Arizona, Tucson, AZ 85721, USA}

\begin{abstract} 
We show that Majorana neutrino flavor mixing can be driven by transition dipole moments in the presence of external electromagnetic fields. We demonstrate the sensitivity of the rotation mixing matrix to strong fields obtaining effective mass eigenstates in the two-flavor model.\footnote{Contribution to the book edited by Gerhard Buchalla, Dieter L\"ust
and Zhi-Zhong Xing \textbf{dedicated to memory of Harald Fritzsch}}
\end{abstract}

\markboth{Johann Rafelski, Andrew Steinmetz, and Cheng Tao Yang}{Neutrinos in EM fields} % Customized running heads

\body

%\tableofcontents\

%%%%%%%%%%%%%%%%%%%%%%%%%%%%%%%%%%%%
\section{Introduction}
\label{sec:intro}
%%%%%%%%%%%%%%%%%%%%%%%%%%%%%%%%%%%%

Among frontiers of research in particle and nuclear physics, the understanding of neutrino physical properties attracts much interest today: Neutrinos are very abundant in the Universe, they were a dominant form of energy density in the Universe for much of its history, and they influence stellar and supernova evolution. As neutrinos are naturally massless in the standard model, the observed flavor oscillation signal non-vanishing neutrino mass. This suggests that neutrinos could provide a window to explore Beyond Standard Model (BSM) physics. This also motivates intense efforts to determine whether they are Dirac-type or Majorana-type fermions with the latter serving as their own antiparticle.

We study the connection between Majorana neutrino transition magnetic dipole moments~\cite{Fujikawa:1980yx,Shrock:1980vy,Shrock:1982sc} and neutrino flavor oscillation. Neutrino electromagnetic (EM) properties have been considered before~\cite{Schechter:1981hw,Giunti:2014ixa,Chukhnova:2019oum,Popov:2019nkr} including the effect of oscillation~\cite{Pal:1991pm,Elizalde:2004mw}. The case of transition moments has the mathematical characteristics of an off-diagonal mass which is distinct from normal direct dipole moment behavior. EM field effects are also distinct from remixing within matter, {\it i.e.\/} the Mikheyev-Smirnov-Wolfenstein effect~\citep{Wolfenstein:1977ue,Mikheyev:1985zog,Smirnov:2003da} which originates from the weak interaction.

We study for the case of two nearly degenerate neutrinos and their eigenstates in the presence of EM fields. We obtain an EM-mass basis distinct from flavor and free-particle mass which mixes flavors as a function of EM fields. Moreover we show solutions relating to full EM field tensor which result in covariant expressions allowing for both magnetic and electrical fields which we have not seen considered before. As neutrinos are electrically neutral, they have no intrinsic magnetic moment due to their spin, therefore any nonzero dipole moment is anomalous. 

An anomalous magnetic moment (AMM) can be introduced into the neutrino effective Lagrangian via a Pauli term~\cite{Steinmetz:2018ryf,Itzykson:1980rh,Schwartz:2014sze}. Noteworthy for Majorana neutrinos, they can only possess transition magnetic moments which couple different flavors electromagnetically do not violate CPT symmetry. However, they break lepton number conservation. Therefore neutrino flavors could be remixed when exposed to strong EM fields.

The size of the neutrino magnetic dipole moment can be constrained as follows: The lower bound is found by higher order standard model interactions with the minimal extension of neutrino mass $m_{\nu}$ included~\cite{Fujikawa:1980yx,Shrock:1980vy,Shrock:1982sc}. The upper bound is derived from reactor, solar and astrophysical experimental observations~\cite{Giunti:2015gga,Canas:2015yoa,Studenikin:2016ykv,AristizabalSierra:2021fuc}. The bounds are expressed in terms of the electron Bohr magneton $\mu_{B}$ as
\begin{align}
\label{bound:1}
\frac{e\hbar G_{F}m_{\nu}c^{2}}{8\pi^{2}\sqrt{2}} \sim 10^{-20}\mu_{B}<\mu_{\nu}^\mathrm{eff}<10^{-10}\mu_{B}\,,\qquad\mu_{B}=\frac{e\hbar}{2m_{e}}
\end{align}
where $G_{F}$ is the Fermi constant and $\mu_{\nu}^\mathrm{eff}$ is the effective and characteristic size of the neutrino magnetic moment. In \req{bound:1}, the lower bound was estimated using a characteristic mass of $m_{\nu}\sim0.1~\mathrm{eV}$. From cosmological studies, the sum of neutrino masses is estimated~\cite{Planck:2018vyg} to be $\sum_{i}m_{i}<0.12$~eV with the effective electron (anti)neutrino mass bounded~\cite{KATRIN:2021uub} by $m_{\nu_{e}}<0.8$~eV.

%%%%%%%%%%%%%%%%%%%%%%%%%%%%%%%%%%%%%%%
\section{Neutrino flavor mixing and electromagnetic fields}
\label{sec:nuflavor}
%%%%%%%%%%%%%%%%%%%%%%%%%%%%%
Oscillation of neutrino flavors observed in experiment is in general interpreted as being due to a difference in neutrino mass and flavor eigenstates. This misalignment between the two representations is described as rotation of the neutrino flavor $N$-vector where $N=3$ is the observed number of generations. The unitary mixing matrix $V_{\ell k}$ allows for the change of basis between mass $(k)$ and flavor $(\ell)$ eigenstates via the transform 
\begin{alignat}{1}
\label{basis:1} \nu_{\ell}=\sum_{k=1}^{3}V_{\ell k}\nu_{k}\,\rightarrow
\begin{pmatrix}
\nu_{e}\\
\nu_{\mu}\\
\nu_{\tau}
\end{pmatrix}=
\begin{pmatrix}
V_{e1} & V_{e2} & V_{e3}\\
V_{\mu1} & V_{\mu2} & V_{\mu3}\\
V_{\tau1} & V_{\tau2} & V_{\tau3}
\end{pmatrix}
\begin{pmatrix}
\nu_{1}\\
\nu_{2}\\
\nu_{3}
\end{pmatrix}\,,
\end{alignat}
where $\nu_{\ell}$ is the neutrino state four-spinor written in the flavor basis while in the mass basis we use $\nu_{k}$ with $k\in1,2,3$. Hereafter we will use implied summation over repeated flavor indices. Spinor indices will be suppressed.

The parameterization of the components of the mixing matrix depends on the Dirac or Majorana-nature of the neutrinos. First let us we recall $U_{\ell k}$, the Dirac neutrino mixing matrix in the standard parameterization~\citep{Schwartz:2014sze} 
\begin{alignat}{1}
\label{rotation:1} U_{\ell k} =
\begin{pmatrix}
 c_{12}c_{13} & s_{12}c_{13} & s_{13}e^{-i\delta}\\
 -s_{12}c_{23} - c_{12}s_{13}s_{23}e^{i\delta} & c_{12}c_{23} - s_{12}s_{13}s_{23}e^{i\delta} & c_{13}s_{23}\\
 s_{12}s_{23} - c_{12}s_{13}c_{23}e^{i\delta}& -c_{12}s_{23} - s_{12}s_{13}c_{23}e^{i\delta} & c_{13}c_{23}
\end{pmatrix}\,,
\end{alignat}
where $c_{ij} = \mathrm{cos}(\theta_{ij})$ and $s_{ij} = \mathrm{sin}(\theta_{ij})$. In this convention, the three mixing angles $(\theta_{12}, \theta_{13}, \theta_{23})$ are understood to be the Euler angles for generalized rotations and $\delta$ is the CP-violating complex phase. 

For the Majorana case we must allow a greater number of complex phases: Majorana neutrinos allow up to two additional complex phases $\rho$ and $\sigma$ which along with $\delta$ participate in CP-violation. A parameterization is achieved by introducing an additional phase matrix $P_{kk'}$
\begin{alignat}{1}
\label{phases:1} &V_{\ell k} = U_{\ell k'}P_{k'k}\,,\\
\label{phases:3} &P_{kk'} = \mathrm{diag}(e^{i\rho},e^{i\sigma},1)\,.
\end{alignat}
The mixing matrix $V_{\ell k}$ defined in \req{phases:1} can then be used to transform the mass matrix $M_{\ell\ell'}$ from the flavor basis into the diagonal mass basis 
\begin{align}
\label{diag:1}
V_{\ell k}^{T}M_{\ell\ell'}V_{\ell'k'} = M_{kk'} = m_{k}\delta_{kk'} = \mathrm{diag}(m_{1},m_{2},m_{3})\,.
\end{align}
We note that there are many interesting models for mass matrices which were pioneered by Fritzsch and Xing~\cite{Fritzsch:1995dj,Fritzsch:1998xs,Fritzsch:1999ee} in the leptonic sector. The masses $m_{k}$ are taken to be real and positive labelling the free propagating states of the three neutrinos.

Turning now to the action, the Majorana mass term in the Lagrangian can be written in the flavor basis as
\begin{alignat}{1}
\label{mass:1} -\mathcal{L}_{\mathrm{mass}}^{\mathrm{Maj.}}=\frac{1}{2}\bar\nu_{\ell}M_{\ell\ell'}\nu_{\ell'}\,,\qquad
M_{\ell\ell'}^{T}=M_{\ell\ell'}\,,
\end{alignat}
where the Majorana fields are written as $\nu=\nu_{L}+C(\bar\nu_{L})^{T}$ and $\bar\nu=\nu^{\dag}\gamma_{0}$ is the Dirac adjoint. The field $\nu_{L}$ refers to left-handed chiral four-component spinors. The charge conjugation matrix $C$ is defined in the usual way in Ref.~(\citeauthor{Itzykson:1980rh}), p.692. Charged conjugated fields are written as $\nu^{c}=C(\bar\nu)^{T}$. We note the Majorana mass matrix is symmetric due to the anticommuting nature of the neutrino fields $\bar\nu\nu=-\nu^{T}\bar\nu^{T}$ and is in general complex~\cite{Adhikary:2013bma,giunti2007fundamentals} though it will be taken to be fully real in this work.

Given these conventions, we can extend our consideration to include the electromagnetic interaction of neutrinos which is possible if neutrinos are equipped with a magnetic moment matrix $\mu_{\ell\ell'}$. We allow for a fixed \emph{external} electromagnetic field tensor $F^{\alpha\beta}$ with a prescribed four-potential $A^{\alpha}_\mathrm{ext}(x^{\mu})$ which imparts a force on the neutrino fields. We emphasize that $F^{\alpha\beta}$ is not dynamical in our formulation and consists of real functions over four-position and not field operators.

We generalize the AMM Pauli spin-field Lagrangian to account for the Majorana fields in the flavor basis following the approach of Ref.~(\citeauthor{Veltman:1997am})
\begin{align}
\label{moment:1}
-\mathcal{L}_{\mathrm{AMM}}^\mathrm{Maj.}=\frac{1}{2}\bar\nu_{\ell}\left(\mu_{\ell\ell'}\frac{1}{2}\sigma_{\alpha\beta}F^{\alpha\beta}\right)\nu_{\ell'}\,.
\end{align}
The operator $\sigma_{\alpha\beta}$ is the $4\times 4$ spin tensor defined by the commutator of the gamma matrices. We would like to point out some interesting features of the Pauli term most notably that the spin tensor itself is not Hermitian with
\begin{align}
\label{notherm:1}
\sigma_{\alpha\beta}^{\dag} = \gamma_{0}\sigma_{\alpha\beta}\gamma_{0}\,.
\end{align}
Despite this, the conjugate of the Lagrangian term in \req{moment:1} is
\begin{align}
\left(\nu^{\dag}\gamma_{0}\sigma_{\alpha\beta}F^{\alpha\beta}\nu\right)^{\dag} = \nu^{\dag}\sigma_{\alpha\beta}^{\dag}F^{\alpha\beta}\gamma_{0}\nu = \nu^{\dag}\gamma_{0}\sigma_{\alpha\beta}F^{\alpha\beta}\nu\,,
\end{align}
which is Hermitian. More about the spin tensor's properties will be elaborated on in \rsec{sec:numoment}.

The Majorana magnetic moment matrix acts in flavor space. It satisfies the following constraints~\cite{Giunti:2014ixa} for CPT symmetry reasons and the anticommuting nature of fermions
\begin{alignat}{1}
\label{props:1}
\mu_{\ell\ell'}^{\dag}=\mu_{\ell\ell'}\,,\qquad
\mu_{\ell\ell'}^{T}=-\mu_{\ell\ell'}\,,
\end{alignat}
{\it i.e.\/} the AMM matrix $\mu_{\ell\ell'}$ is Hermitian and fully anti-symmetric. This requires that the transition magnetic moment elements are purely imaginary while all diagonal AMM matrix elements vanish
\begin{align}
\label{mu:1}
\mu_{\ell\ell'}=
\begin{pmatrix}
\mu_{ee} & \mu_{e\mu} & \mu_{e\tau} \\
\mu_{\mu e} & \mu_{\mu\mu} & \mu_{\mu\tau} \\
\mu_{\tau e} & \mu_{\tau\mu} & \mu_{\tau\tau}
\end{pmatrix}\xrightarrow{\mathrm{Majorana}}
\mu_{\ell\ell'}=
\begin{pmatrix}
0 & i\mu_{e\mu} & -i\mu_{e\tau} \\
-i\mu_{e\mu} & 0 & i\mu_{\mu\tau} \\
i\mu_{e\tau} & -i\mu_{\mu\tau} & 0
\end{pmatrix}\,.
\end{align}

We can combine the mass term in~\req{mass:1} and AMM contribution in~\req{moment:1} into a single effective Lagrangian
\begin{align}
\label{massmom:1}
\mathcal{L}_\mathrm{eff}^\mathrm{Maj.} &= \mathcal{L}_\mathrm{kinetic}^\mathrm{Maj.} + \mathcal{L}_\mathrm{mass}^\mathrm{Maj.} + \mathcal{L}_\mathrm{AMM}^\mathrm{Maj.}\,,\\
\label{massmom:2}
\mathcal{L}_\mathrm{eff}^\mathrm{Maj.} &= \mathcal{L}_\mathrm{kinetic}^\mathrm{Maj.} - \frac{1}{2}\bar\nu_{\ell}\left(M_{\ell\ell'}+\mu_{\ell\ell'}\frac{1}{2}\sigma_{\alpha\beta}F^{\alpha\beta}\right)\nu_{\ell'}\;.
\end{align}
\req{massmom:2} is our working Lagrangian. For later convenience we define the generalized mass-dipole matrix $\mathcal{M}_{\ell\ell'}$ present in \req{massmom:2} as
\begin{align}
\label{massmom:3}
\mathcal{M}_{\ell\ell'}(E,B)\equiv M_{\ell\ell'}+\mu_{\ell\ell'}\frac{1}{2}\sigma_{\alpha\beta}F^{\alpha\beta}\,,\qquad \mathcal{M}_{\ell\ell'}^{\dag}=\gamma_{0}\mathcal{M}_{\ell\ell'}\gamma_{0}\,.
\end{align}
As neutrinos must propagate as energy eigenstates, our objective is to find the eigenvalues of \req{massmom:2} rather than \req{diag:1}. As the mass eigenvalues are modified by the presence the EM interactions $m\rightarrow\widetilde m(E,B)$ so will the mixing matrix, leading to modifications of \req{phases:1}. These electromagnetic components then facilitate time-dependant oscillation among the free-particle mass eigenstates~\cite{Giunti:2014ixa}.

%%%%%%%%%%%%%%%%%%%%%%%%%%%%%%%%%%%%%%%
\section{Properties of Pauli coupling to EM fields}
\label{sec:numoment}
%%%%%%%%%%%%%%%%%%%%%%%%%%%%%%%%%%%%%%%
The electromagnetic dipole behavior of the neutrino depends on mathematical properties of the tensor product $\sigma_{\alpha\beta}F^{\alpha\beta}$. We prefer to work in the Weyl (chiral) spinor representation where the EM contribution is diagonal in spin space. Therefore we evaluate the product $\sigma_{\alpha\beta}F^{\alpha\beta}$ in the chiral representation following Feynman and Gell-mann\cite{Feynman:1958ty} yielding
\begin{align}
\label{chiral:1}
-\frac{1}{2}\sigma_{\alpha\beta}F^{\alpha\beta}=
\begin{pmatrix}
\vec{\sigma}\cdot(\vec{B}+i\vec{E}/c) & 0\\
0 & \vec{\sigma}\cdot(\vec{B}-i\vec{E}/c)
\end{pmatrix}\equiv
\begin{pmatrix}
\vec{\sigma}\cdot\vec{f}_{+} & 0 \\
0 & \vec{\sigma}\cdot\vec{f}_{-}
\end{pmatrix}\,,
\end{align}
where we have defined the complex electromagnetic field $\vec{f}_{\pm}=\vec{B}\pm i\vec{E}/c$. As \req{chiral:1} is diagonal in this representation, it does not exchange handedness when acting upon a state. Since left and right-handed neutrinos are not remixed by transition moments, we continue to disregard presence of any right-handed neutrinos. \req{chiral:1} shows directly that neutrinos are sensitive to both magnetic and electric fields with the latter often neglected in theoretical neutrino studies. We can also see in \req{chiral:1} explicitly its non-Hermitian character, see \req{massmom:2}, of the EM spin-field coupling. Specifically this is mirrored in $\vec{f}_{\pm}$ as $(\vec{f}_{\pm})^{*}=\vec{f}_{\mp}$: The complex EM fields have a Hermitian $(\vec{B})$ and anti-Hermitian $(i\vec{E})$ part.

For later convenience we note how the EM invariants $\mathcal{S}$ and $\mathcal{P}$ can help understand the AMM term 
\begin{align}
\label{invar:1}
\left(\frac{1}{2}\sigma_{\alpha\beta}F^{\alpha\beta}\right)^{2}=
\begin{pmatrix}
\mathcal{S}+i\mathcal{P} & 0\\
0 & \mathcal{S}-i\mathcal{P}
\end{pmatrix}=\mathcal{S}-i\gamma_{5}\mathcal{P}\,,\\
\mathcal{S}\equiv\frac{1}{2}\left(B^{2}-E^{2}/c^{2}\right)\,,\qquad
\mathcal{P}\equiv\vec{B}\cdot\vec{E}/c\,,\qquad
\frac{1}{2}\vec{f}_{\pm}\cdot\vec{f}_{\pm}=\mathcal{S}\pm i\mathcal{P}\,.
\end{align}
The combination of these invariants make up the eigenvalues of the Pauli term. Moreover, taking the product of $\vec{f}_{\pm}$ with its complex conjugate we find
\begin{align}
\label{cross:1}
\frac{1}{2}\left(\vec{\sigma}\cdot\vec{f}_{\pm}\right)\left(\vec{\sigma}\cdot\vec{f}_{\mp}\right)=T^{00}\mp \sigma_{i}T^{0i}\,,
\end{align}
where we recognize the stress-energy tensor $T^{\alpha\beta}$ component $T^{00}$ for field energy density and $T^{0i}$ momentum density respectively
\begin{align}
T^{00}=\frac{1}{2}\left(B^{2}+E^{2}/c^{2}\right)\,,\qquad
T^{0i}=\frac{1}{c}\varepsilon_{ijk}E_{j}B_{k}\,.
\end{align}
As we will see in \rsec{sec:toy}, \req{cross:1} will appear in the EM-mass eigenvalues of our effective Lagrangian \req{massmom:1}. Using the identity in \req{chiral:1} and \req{cross:1} we also find the interesting relationship
\begin{align}
\label{cross:2}
\frac{1}{2}\left(\frac{1}{2}\sigma_{\alpha\beta}F^{\alpha\beta}\right)\left(\frac{1}{2}\sigma_{\alpha\beta}F^{\alpha\beta}\right)^{\dag}=
\gamma_{0}\left(T^{00}\gamma_{0}+T^{0i}\gamma_{i}\right)\,.
\end{align}
Now that we have elaborated on the relevant EM field identities, we turn back to the magnetic dipole and flavor rotation problem.

%%%%%%%%%%%%%%%%%%%%%%%%%%%%%%%%%%%%%%%
\section{Toy model: EM-flavor mixing for two generations with a real Hermitian mass matrix}
\label{sec:toy}
%%%%%%%%%%%%%%%%%%%%%%%%%%%%%%%%%%%%%%%
Considering experimental data on neutrino oscillations, it is understood that either the two heavier (normal hierarchy) or the two lighter (inverted hierarchy) neutrino states are close together in mass. If the electromagnetic properties of the neutrino do indeed lead to flavor mixing effects, then it is likely the closer pair of neutrino mass states that are most sensitive to the phenomenon we explore. In the spirit of Bethe~\cite{Bethe:1986ej}, we therefore explore the $N=2$ two generation $(\nu_{e},\nu_{\mu})$ model.

Following the properties established in \req{props:1} and \req{massmom:3} we write down the two generation mass and dipole matrices as
\begin{alignat}{1}
\label{mix:1} M_{\ell\ell'}= 
\begin{pmatrix}
m_{\nu_{e}} & {\delta m}\\
{\delta m} & m_{\nu_{\mu}}
\end{pmatrix}\,,\qquad
\mu_{\ell\ell'} = 
\begin{pmatrix}
0 & i\delta\mu\\
-i\delta\mu & 0
\end{pmatrix}\,.
\end{alignat}
The AMM coupling $\delta\mu$ is taken to be real with a pure imaginary coefficient. While the mass elements $(m_{\nu_{e}},m_{\nu_{\mu}},{\delta m})$ are generally complex, we choose in our toy model for them to be fully real
\begin{align}
\label{choice:1}
m_{\nu_{e}}=m_{\nu_{e}}^{*}\,,\qquad
m_{\nu_{\mu}}=m_{\nu_{\mu}}^{*}\,,\qquad
\delta m=\delta m^{*}\,,
\end{align}
making the mass matrix $M_{\ell\ell'}$ Hermitian. This allows us to more easily evaluate and emphasize the EM contributions to mixing rather than complications arising from the mass matrix.

Using \req{mix:1} and \req{choice:1}, we write the mass-dipole matrix in \req{massmom:3} in terms of $2\times2$ flavor components as
\begin{align}
\label{mix:2}
\mathcal{M}_{\ell\ell'} = 
\begin{pmatrix}
m_{\nu_{e}} & {\delta m}+i\delta\mu\sigma_{\alpha\beta}F^{\alpha\beta}/2\\
{\delta m}-i\delta\mu\sigma_{\alpha\beta}F^{\alpha\beta}/2 & m_{\nu_{\mu}}
\end{pmatrix}\,,\qquad
\mathcal{M}_{\ell\ell'}^{\dag}=\gamma_{0}\mathcal{M}_{\ell\ell'}\gamma_{0}\,.
\end{align}
As noted before, this matrix is not Hermitian due to the inclusion of the spin tensor, therefore it is not guaranteed to satisfy an algebraic eigenvalue equation in its present form which is a requirement for well behaved masses.

This can be remedied by recalling that any arbitrary complex matrix can be diagonalized into its real eigenvalues $\lambda_{j}$ by the biunitary transform
\begin{align}
\label{biunitary:1}
W_{\ell j}^{\dag}\mathcal{M}_{\ell\ell'}Y_{\ell'j'}=\lambda_{j}\delta_{jj'}\,,
\end{align}
where $Y_{\ell j}$ and $W_{\ell j}$ are both unitary matrices. Taking the complex conjugate of \req{biunitary:1}, we arrive at
\begin{align}
\label{biunitary:2}
(W_{\ell j}^{\dag}\mathcal{M}_{\ell\ell'}Y_{\ell'j'})^{\dag} = 
Y_{\ell j'}^{\dag}\gamma_{0}\mathcal{M}_{\ell\ell'}\gamma_{0}W_{\ell' j}=\lambda_{j}\delta_{jj'}\,,\\
Y_{\ell j}=\gamma_{0}W_{\ell j}\rightarrow
W_{\ell j}^{\dag}\mathcal{M}_{\ell\ell'}\gamma_{0}W_{\ell'j'}=\lambda_{j}\delta_{jj'}\,. 
\end{align}
As $Y_{\ell j}$ and $W_{\ell j}$ are related by a factor of $\gamma_{0}$ based on the conjugation properties of \req{mix:2}, this lets us eliminate $Y_{\ell j}$ and diagonalize using a single unitary matrix $W_{\ell j}$. The related matrix $\mathcal{M}_{\ell\ell'}\gamma_{0}$ is Hermitian
\begin{align}
\label{herm:1}
(\mathcal{M}_{\ell\ell'}\gamma_{0})^{\dag} = \mathcal{M}_{\ell\ell'}\gamma_{0}\,,
\end{align}
and also equivalent to the root of the Hermitian product of \req{mix:2}
\begin{align}
(\mathcal{M}\mathcal{M}^{\dag})_{\ell\ell'} = \left((\mathcal{M}\gamma_{0})(\mathcal{M}\gamma_{0})\right)_{\ell\ell'}\,.
\end{align}
Therefore a suitable unitary transformation $W_{\ell j}$ rotates flavor $\ell$-states into magnetized mass $j$-states. The eigenvalues $\lambda_{j}^{2}$ of $(\mathcal{M}\mathcal{M}^{\dag})_{\ell\ell'}$ are the squares of both signs of the eigenvalues of $\mathcal{M}_{\ell\ell'}\gamma_{0}$. We write this property (with flavor indices suppressed) as
\begin{align}
W^{\dag}(\mathcal{M}\mathcal{M}^{\dag})W &= W^{\dag}(\mathcal{M}\gamma_{0})WW^{\dag}(\mathcal{M}\gamma_{0})W = \mathrm{diag}(\lambda_{1}^{2},\lambda_{2}^{2})\,.
\end{align}
We associate $\lambda_{j}=\widetilde m_{j}(E,B)$ with $j\in1,2$ as the effective EM-mass states which are field dependant in this basis. 

The matrix $W_{\ell j}$ mixes flavor states in a manner distinct from the free-particle case. We proceed to evaluate $W_{\ell j}$ breaking the rotation into two separate unitary transformations: a) $(\ell\rightarrow k)$ rotation $V_{\ell k}$ to free-particle mass; and b) $(k\rightarrow j)$ rotation $Z_{kj}$ to the EM-mass basis. Guided by \req{basis:1} we write
\begin{align}
\label{zrot:1}
\nu_{j} = W^{\dag}_{\ell j}\nu_{\ell} = Z_{kj}^{\dag}V_{\ell k}^{\dag}\nu_{\ell}\,.
\end{align}
In the limit that the EM fields go to zero, the magnetized rotation becomes unity $Z_{kj}\rightarrow\delta_{kj}$ thereby ensuring the EM-mass basis and free-particle mass basis become equivalent. The rotation $Z_{kj}$ can then be interpreted as the external field forced rotation.

According to \req{diag:1}, the mass matrix in \req{mix:1} can be diagonalized in the two generation case by a one parameter unitary mixing matrix $V_{\ell k}$ given by
\begin{align}
\label{rot:1}
V_{\ell k}(\theta)=
\begin{pmatrix}
\cos\theta & \sin\theta\\
-\sin\theta & \cos\theta
\end{pmatrix}\,.
\end{align}
For a real Hermitian $2\times 2$ mass matrix, the rotation matrix $V_{\ell k}$ is real and only depends on the angle $\theta$. The eigenvalues of the original Hermitian mass matrix are given by
\begin{align}
\label{massroot:1}
m_{1,2}=\frac{1}{2}\left(m_{\nu_{e}}+m_{\nu_{\mu}}\mp\sqrt{|\Delta m_{0}|^{2}+4\delta m^{2}}\right)\,,\qquad
|\Delta m_{0}|=|m_{\nu_{\mu}}-m_{\nu_{e}}|\,.
\end{align}
We assign $m_{1}$ to the lower mass $(-)$ root and $m_{2}$ with the larger mass $(+)$ additive root. The rotation $\theta$ in \req{rot:1} is then given by
\begin{align}
\label{massroot:2}
\sin2\theta=\sqrt{\frac{4\delta m^{2}}{|\Delta m_{0}|^{2}+4\delta m^{2}}}\,,\qquad
\cos2\theta=\sqrt{\frac{|\Delta m_{0}|^{2}}{|\Delta m_{0}|^{2}+4\delta m^{2}}}\,.
\end{align}



In our toy model, the off-diagonal imaginary transition magnetic moment  $\mu_{\ell\ell'}$ commutes with the real valued mixing matrix $V_{\ell k}$ and the following relations hold
\begin{align}
\label{commuting:1}
V_{\ell k}^{\dag}\mu_{\ell\ell'}V_{\ell' k'}=(V^{\dag}V)_{k\ell'}\mu_{\ell'k'}=\mu_{kk'}=
\begin{pmatrix}
0 & i\delta\mu\\
-i\delta\mu & 0
\end{pmatrix}\,.
\end{align}
We see that the Majorana transition dipoles in our model are off-diagonal in both flavor and mass basis. This is an important property because it reveals that the real parameter unitary matrix in \req{commuting:1} cannot rotate a pure imaginary matrix at least in the two flavor case. We apply the rotation in \req{rot:1} to \req{herm:1} yielding
\begin{align}
\label{herm:2}
V_{\ell k}^{\dag}(\mathcal{M}_{\ell\ell'}\gamma_{0})V_{\ell' k'} &= 
V_{\ell k}^{\dag}M_{\ell\ell'}\gamma_{0}V_{\ell' k'} +
V_{\ell k}^{\dag}(\mu_{\ell\ell'}\sigma_{\alpha\beta}\gamma_{0}F^{\alpha\beta}/2)V_{\ell' k'}\,,\\
\label{herm:3}
V_{\ell k}^{\dag}(\mathcal{M}_{\ell\ell'}\gamma_{0})V_{\ell' k'} &= 
\begin{pmatrix}
m_{1}\gamma_{0} & i\delta\mu\sigma_{\alpha\beta}\gamma_{0}F^{\alpha\beta}/2\\
-i\delta\mu\sigma_{\alpha\beta}\gamma_{0}F^{\alpha\beta}/2 & m_{2}\gamma_{0}
\end{pmatrix}\equiv
\begin{pmatrix}
\mathcal{A} & i\mathcal{C}\\
-i\mathcal{C} & \mathcal{B}
\end{pmatrix}\,,
\end{align}
where we have defined implicitly the Hermitian elements $(\mathcal{A},\mathcal{B},\mathcal{C})$.  Applying now both rotations 
to \req{herm:1} yields
\begin{align}
\label{herm:4}
W_{\ell j}^{\dag}(\mathcal{M}_{\ell\ell'}\gamma_{0})W_{\ell' j'} &= 
Z_{kj}^{\dag}\begin{pmatrix}
\mathcal{A} & i\mathcal{C}\\
-i\mathcal{C} & \mathcal{B}
\end{pmatrix}Z_{k'j'}=\lambda_{j}\delta_{jj'}\,.
\end{align}
\req{herm:4} can be understood as an eigenvalue equation where the columns of $W_{\ell j}$ are interpreted as eigenvectors for each eigenvalue $\lambda_{j}$.

The explicit form of the EM-field related rotation $Z_{kj}$ introduced in \req{zrot:1} is
\begin{align}
\label{zrot:2}
Z_{kj}(\omega,\phi)=
\begin{pmatrix}
\cos\omega & e^{i\phi}\sin\omega\\
-e^{-i\phi}\sin\omega & \cos\omega
\end{pmatrix}\,,\qquad
W_{\ell j}(\theta,\omega,\phi)=V_{\ell k}(\theta)Z_{kj}(\omega,\phi)\,,
\end{align}
where $Z_{kj}$ depends on the real angle $\omega$ and complex phase $\phi$. The full rotation $W_{\ell j}$ therefore depends on three parameters when broken into free-particle rotation and EM rotation. The effective EM-mass eigenstates are then solutions to the characteristic polynomial
\begin{align}
\label{poly:1}
(\mathcal{A}-\lambda_{j}\gamma_{0})(\mathcal{B}-\lambda_{j}\gamma_{0})-\mathcal{C}^{2}=0\,,
\end{align}
which we obtained by taking the determinant of \req{herm:3} over flavor but not spin space. It is useful to define the following identities for the off-diagonal element
\begin{align}
\label{poly:1a}
\mathcal{C}^{2} = 
\delta\mu^{2}\left(\frac{1}{2}\sigma_{\alpha\beta}F^{\alpha\beta}\right)\left(\frac{1}{2}\sigma_{\alpha\beta}F^{\alpha\beta}\right)^{\dag}=
2\delta\mu^{2}\gamma_{0}\left(T^{00}\gamma_{0}+T^{0i}\gamma_{i}\right)\,,
\end{align}
and for the diagonal elements
\begin{align}
(\mathcal{B}-\mathcal{A})^{2} = |m_{2}-m_{1}|^{2} = |\Delta m|^{2}\,,\qquad (\mathcal{A}+\mathcal{B})\gamma_{0} = m_{1} + m_{2}\,.
\end{align}
\req{poly:1a} was obtained using the expression in \req{cross:2}. Because of the spinor behavior of each element, the eigenvalues are obtained with $\gamma_{0}$ coefficients. \req{poly:1} therefore has the roots $\lambda_{1,2} = \widetilde m_{1,2}(E,B)$
\begin{align}
\label{poly:2}
\widetilde m_{1,2}(E,B)\! &=\! \frac{1}{2}\left(m_{1}\!+\!m_{2}\!\mp\!\sqrt{|\Delta m|^{2}\!+\!8\delta\mu^{2}\gamma_{0}\left(T^{00}\gamma_{0}+T^{0i}\gamma_{i}\right)}\right)\!,\\
\label{poly:3}
\widetilde m_{1,2}(E,B)\! &=\! \frac{1}{2}\left(m_{1}\!+\!m_{2}\!\mp\!\sqrt{|\Delta m|^{2}\!+\!8\delta\mu^{2}\gamma_{0}\left(\gamma_{0}\frac{1}{2}\left(B^{2}\!+\!\frac{E^{2}}{c^{2}}\right)\!+\!\vec{\gamma}\!\cdot\!(\frac{\vec{E}}{c}\times\vec{B})\right)}\right)\!.
\end{align}
The EM-mass eigenstates $\widetilde m(E,B)$ depends on the energy density $T^{00}$ of the EM field and the spin projection along the EM momentum density $T^{0i}$. However the coefficient $\delta\mu^{2}$ is presumed to be very small, therefore the EM contribution only manifests in strong EM fields or where the free-particle case has very nearly or exactly degenerate masses. When the the electromagnetic fields go to zero, the EM-masses in \req{poly:3} reduce to the free-particle result printed in \req{massroot:1} as expected.

The complex phase in \req{zrot:2} has the value $\phi=\pi(n-1/2)$ with $n\in0,\pm1,\pm2...$ making the complex exponential in \req{zrot:2} pure imaginary. Complex phases in mixing matrices are generally associated with CP-violation. Therefore, this suggests that CP violation in the neutrino sector can be induced in the presence of external background fields which lift symmetries.

We note that the solution in \req{poly:3} actually contain four distinct EM-mass eigenstates $\widetilde m_{j}^{s}(E,B)$ with the lower $(j=1)$ and upper $(j=2)$ masses and the additional spin splitting from the alignment $(s=+1)$ or anti-alignment $(s=-1)$ of the neutrino spin with the momentum density of the external EM field. For good spin eigenstates $s\in\pm1$, we can rewrite \req{poly:1a} with EM fields explicitly as
\begin{align}
\label{spinsplit:1}
\mathcal{C}^{2}_{s}=2\delta\mu^{2}\left(\frac{1}{2}(B^{2}+E^{2}/c^{2})+s|\vec{E}/c\times\vec{B}|\right)\,.
\end{align}
The above expression within the square is positive definite, therefore \req{spinsplit:1} is always real. Spin splitting requires that we consider separate rotations for each spin state as the rotation angle $\omega_{s}$ depends on the spin quantum number
\begin{align}
\label{zrot:3}
\sin2\omega_{s}=\sqrt{\frac{4\mathcal{C}_{s}^{2}}{|\Delta m|^{2}+4\mathcal{C}_{s}^{2}}}\,,\qquad
\cos2\omega_{s}=\sqrt{\frac{|\Delta m|^{2}}{|\Delta m|^{2}+4\mathcal{C}_{s}^{2}}}\,.
\end{align}
Spin splitting vanishes for the pure electric or magnetic field cases. The rotation in \req{zrot:3} is mathematically similar to that of the free-particle case written in \req{massroot:2} in the two flavor generation model.

The rotation angles in \req{zrot:3} reveal two distinct limits where EM-masses are dominated by either: a) the intrinsic mass splitting $\mathcal{C}_{s}\ll|\Delta m|^{2}$ with $\omega_{s}\rightarrow0$ or b) the EM contribution $\mathcal{C}_{s}\gg|\Delta m|^{2}$ where $\omega_{s}\rightarrow\pi/4$.
For the first case where the masses are not degenerate or the fields are weak, we obtain the expansion
\begin{align}
\label{series:1}
\lim_{\mathcal{C}_{s}\ll|\Delta m|^{2}}\widetilde m_{1,2}^{s}(E,B)=\frac{1}{2}\left(m_{1}+m_{2}\mp|\Delta m|\left(1+\frac{2\mathcal{C}_{s}^{2}}{|\Delta m|^{2}}+\ldots\right)\right)\,,
\end{align}
which as stated before reduces to the free-particle case at lowest order.

In the opposite limit, where the masses are very nearly degenerate or fields are strong, the EM-mass eigenvalues in \req{poly:3} can be approximated by the series
\begin{align}
\label{series:2}
\lim_{\mathcal{C}_{s}\gg|\Delta m|^{2}}\widetilde m_{1,2}^{s}(E,B)=\frac{1}{2}\left(m_{1}+m_{2}\mp2\mathcal{C}_{s}\left(1+\frac{|\Delta m|^{2}}{8\mathcal{C}_{s}^{2}}+\ldots\right)\right)
\end{align}
For fully degenerate free-particle masses $m_{1}=m_{2}$, this reduces to
\begin{align}
\label{series:2a}
\lim_{|\Delta m|^{2}\to0}\widetilde m_{1,2}^{s}(E,B)=m_{1}\mp\mathcal{C}_{s}\,.
\end{align}
\req{series:2a} indicates that for degenerate free-particle masses, the effective splitting $|\Delta m_\mathrm{EM}|\equiv\mathcal{C}_{s}$ between masses arises purely from the electromagnetic interaction of the neutrinos.

Because of the bounds in \req{bound:1} on the effective neutrino magnetic moment, we can estimate the field strength required for an external magnetic field to generate an electromagnetic mass splitting of $|\Delta m_\mathrm{EM}|=10^{-3}$~eV. Using the upper limit for the neutrino effective moment of $\mu_{\nu}^\mathrm{eff}\sim10^{-10}\mu_{B}$ we obtain
\begin{align}
\label{estimate:1}
\left.\frac{\mathcal{C}_{s}}{\mu_{\nu}^\mathrm{eff}}\right\rvert_{\vec{E}=0}=\frac{10^{-3}\,\mathrm{eV}}{10^{-10}\mu_{B}}\approx1.7\times10^{11}\,\mathrm{T}\,.
\end{align}
This is near the upper bound of the magnetic field strength of magnetars~\cite{Kaspi:2017fwg} which are of the order $10^{11}$~Tesla. In this situation, the EM contribution rivals the estimated inherent splitting~\cite{ParticleDataGroup:2022pth} of the two closer massive neutrinos. Primordial magnetic fields~\cite{Grasso:2000wj} in the Early Universe may also present an environment for significant EM neutrino flavor mixing as both the external field strength and density of neutrinos would be very large~\cite{Rafelski:2023emw}. The magnetic properties of neutrinos may also have contributed alongside the charged leptons in magnetization in the Early Universe~\cite{Steinmetz:2023nsc} prior to recombination. While the above estimate was done with astrophysical systems in mind, we note that strong electrical fields should also produce EM-mass splitting therefore neutrinos in dense matter environments near nuclei is also of interest. Therefore, if neutrinos have abnormally large transition magnetic dipole moments, then they should exhibit mass splitting from the neutrino's electromagnetic dipole interaction which may compete with the intrinsic mass differences of the free-particles.

%%%%%%%%%%%%%%%%%%%%%%%%%%%%%%%%%%%%
\section{Conclusions}
We have incorporated electromagnetic effects in the Majorana neutrino mixing matrix by introducing an anomalous transition magnetic dipole moment. We have described the formalism for three generations of neutrinos and explicitly explored the two generation case as a toy model. 

In the two generation case, we determined the effect of electric and magnetic fields on flavor rotation in \req{zrot:2} by introducing an electromagnetic flavor unitary rotation $Z_{kj}$. We presented remixed mass eigenstates $\widetilde m(E,B)$ in \req{poly:3} which are the propagating mass-states in a background electromagnetic field. These EM-mass eigenstates were also further split by spin aligned and anti-aligned states relative to the external field momentum density. There is much left to do to explore further the nascent connection between spin and flavor via transition magnetic moments. 

Of particular interest is the case of nearly degenerate free-particle mass eigenstates where the EM effects are most manifest. For nearly degenerate masses compared to EM fields described by $\mathcal{C}_{s}(E,B)\gg|\Delta m|^{2}$ in \req{spinsplit:1}, the mass splitting is dominated by the electromagnetic contribution. We fine this effect would be most relevant for strong magnetic field environments such as around magnetars or primordial magnetic fields in the Early Universe and are capable of competing with the mass splitting seen in the two closer neutrino mass states. We also emphasize that dense matter environments may also be relevant where the electrical field energy density is large.

A further extension of this work would be to include matter remixing as an additional term in the effective Lagrangian in \req{massmom:2}. More speculatively, as transition dipoles act as a mechanism to generate mass via coupling the states to the EM energy density $T^{00}$ as seen in \req{poly:3}, an additional consideration of our work is to postulate the presence of a dark vector field in the Universe which if coupled to neutrinos would behave like off-diagonal masses in flavor. Neutrinos would then have non-zero masses in the Universe by virtue of dipole interactions.

%%%%%%%%%%%%%%%%%%%%%%%%%%%%%%%%%%%%
\section*{Acknowledgements (by Johann Rafelski)}
Harald Fritzsch enjoyed the US-South-West. On the way between Caltech and the Santa Fe Institute he sometimes made a stop at the half-way point, Tucson. On such occasions he explored Arizona mountains and deserts; Figure\,\ref{Fig:AZcolloq2007} is showing our outing on occasion of March 23, 2007 physics colloquium at the University of Arizona: Harald was an avid observer and photographer of the desert fauna and flora. 

\begin{figure}%[hb]
\centerline{\includegraphics[width=0.95\columnwidth]{07March24HaraldCollageDesertMuseum.jpg}}
\caption{Harald Fritzsch visiting Arizona-Sonora Desert Museum in Spring 2007. Pictures and picture assembly by Johann Rafelski
}
\label{Fig:AZcolloq2007} 
\end{figure}

These meetings offered an opportunity to exchange ideas and discuss frontier topics in different areas of physics. Strong interactions, the origin of neutrino mass and the understanding of the parameters of the standard model were close to Harald's heart. Were these parameters really natural constants on cosmological time scale? In Figure\,\ref{Fig:RANP2004} we see Harald's first transparency ``Time Dependence of QCD and Experimental Tests'' made at at the 9th Hadron Physics and 7th Relativistic Aspects of Nuclear Physics (HADRON-RANP 2004): A Joint Meeting on QCD and QGP: Rio de Janeiro, Brazil, March 28-April 3, 2004~\cite{Fritzsch:2004civ}, a meeting we both attended. We see that Harald modified slightly by hand the typed transparency to introduce the meeting specific context in a talk which arose from another publication of the epoch, Ref.~(\citeauthor{Calmet:2001nu}). 

\begin{figure}%[ht]
\centerline{\includegraphics[width=0.95\columnwidth]{04RANPHarald1Ed.jpg}}
\caption{Harald begins his presentation in Rio de Janeiro 2004 about time dependence of QCD, see text for details. Picture by Johann Rafelski
}
\label{Fig:RANP2004} 
\end{figure}

With Harald's passing I lost a friend of more than 40 years, and, equally importantly, a colleague whose fast mind, willingness to listen, and clarity of thought, helped in some of my own challenges. I had very much wished to hear his opinion on this work. 
%%%%%%%%%%%%%%%%%%%%%%%%%%%%%%%%%%%%


\bibliographystyle{ws-rv-van}
\bibliography{Rafelski_Steinmetz_for_Harald}
%%%%%%%%%%%%%%%%%%%%%%%%%%%%%%%%%%%%
\end{document} 
%%%%%%%%%%%%%%%%%%%%%%%%%%%%%%%%%%%%
