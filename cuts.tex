The most general unitary $2\times2$ matrix can be written as
\begin{alignat}{1}
	\label{mix:4} W_{\ell k}(\theta) = 
    \left(
    \begin{array}{cc}
         e^{i \alpha } \cos (\theta ) & e^{i (\beta +\phi )} \sin (\theta ) \\
         -e^{i (\alpha -\phi )} \sin (\theta ) & e^{i \beta } \cos (\theta ) \\
    \end{array}
    \right)
\end{alignat}
which contains up to three phases $\alpha,\beta,\gamma$ and angle $\theta$.




Since transition AMM elements serve to break lepton number conservation allowing for $\nu_{\ell}+\gamma\rightarrow\nu_{\ell'}$ processes, neutrinos could be remixed when exposed to strong EM fields similar to remixing within matter in the Mikheyev-Smirnov-Wolfenstein (MSW) effect~\citep{Wolfenstein:1977ue,Mikheyev:1985zog}.
    
 \begin{align}
    \label{chiral:1b}
   \frac{1}{2}\sigma_{\alpha\beta}F^{\alpha\beta}\psi_{L} = \vec{\sigma}\cdot\vec{f}_{+}\psi_{L}\,,\qquad
    \frac{1}{2}\sigma_{\alpha\beta}F^{\alpha\beta}\psi_{R} = \vec{\sigma}\cdot\vec{f}_{-}\psi_{R}\,,\qquad
\end{align}
where $\psi_{L}$ and $\psi_{R}$ are arbitrary left and right-handed two-component spinors.




At this point it is good to remember that quantum mechanics predicts that relativistic massive neutral particles with magnetic dipoles in magnetic fields have energy eigenstates~\cite{Steinmetz:2018ryf} of
\begin{align}
    \label{kgp:1}
    E(p,B) = \sqrt{m^{2}c^{4}\pm2\mu Bmc^{2}+p^{2}c^{2}}
\end{align}
depending on the spin alignment of the particle with the magnetic field. Therefore neutrinos propagating in EM fields will not be travelling in free-particle mass eigenstates, but magnetized states.




Considering the generalization of the $2\times 2$ Pauli matrices~\cite{Ohlsson:2011zz} we obtain the covariant form 
\begin{align}       \label{cross:1a}    
        \sigma^{\alpha}=(1,+\vec{\sigma})\,,\qquad
        \bar\sigma^{\alpha}=(1,-\vec{\sigma})\,.
\end{align}
which mirrors the Weyl (chiral) representation of $\gamma^\alpha$
\begin{align}       \label{cross:1c}    
\gamma^\alpha =(off-diag I_2,I_2;\vec\sigma,-\vec\sigma)
\end{align}
This allows us to write 
        \begin{align}
        \label{cross:2}        \left(\vec{\sigma}\cdot\vec{f}_{+}\right)\left(\vec{\sigma}\cdot\vec{f}_{-}\right)=\sigma_{\alpha}\sigma_{\beta}T^{\alpha\beta}\,,\qquad
        \left(\vec{\sigma}\cdot\vec{f}_{-}\right)\left(\vec{\sigma}\cdot\vec{f}_{+}\right)=\bar\sigma_{\alpha}\bar\sigma_{\beta}T^{\alpha\beta}\,,
\end{align}
where $\sigma^{\alpha}$ and $\bar\sigma^{\alpha}$ are the Pauli 4-vectors and $T^{\alpha\beta}$ is the EM stress-energy tensor.




We note that the Lagrangian term in \req{massmom:1} is acting to the right on a left-handed chiral neutrino spinor (with the right-handed charged conjugated portion being acted upon in the Hermitian conjugate). Therefore, we can rewrite \req{massmom:1} in two-component form as
\begin{align}
    \label{massmom:1a}
    \mathcal{L}_\mathrm{eff}^\mathrm{Maj.} = 
    -\frac{1}{2}\nu_{L,2,\ell}^{T}C_{2}^{\dag}\left(M_{\ell\ell'}+\mu_{\ell\ell'}\vec{\sigma}\cdot\vec{f}_{+}\right)\nu_{L,2,\ell'}+\mathrm{h.c.}
\end{align}
where the operator $C_{2}$ is the charge conjugation matrix is written as a $2\times 2$ matrix.