The most general unitary $2\times2$ matrix can be written as
\begin{alignat}{1}
	\label{mix:4} W_{\ell k}(\theta) = 
    \left(
    \begin{array}{cc}
         e^{i \alpha } \cos (\theta ) & e^{i (\beta +\phi )} \sin (\theta ) \\
         -e^{i (\alpha -\phi )} \sin (\theta ) & e^{i \beta } \cos (\theta ) \\
    \end{array}
    \right)
\end{alignat}
which contains up to three phases $\alpha,\beta,\gamma$ and angle $\theta$.




 \begin{align}
    \label{chiral:1b}
   \frac{1}{2}\sigma_{\alpha\beta}F^{\alpha\beta}\psi_{L} = \vec{\sigma}\cdot\vec{f}_{+}\psi_{L}\,,\qquad
    \frac{1}{2}\sigma_{\alpha\beta}F^{\alpha\beta}\psi_{R} = \vec{\sigma}\cdot\vec{f}_{-}\psi_{R}\,,\qquad
\end{align}
where $\psi_{L}$ and $\psi_{R}$ are arbitrary left and right-handed two-component spinors.




At this point it is good to remember that quantum mechanics predicts that relativistic massive neutral particles with magnetic dipoles in magnetic fields have energy eigenstates~\cite{Steinmetz:2018ryf} of
\begin{align}
    \label{kgp:1}
    E(p,B) = \sqrt{m^{2}c^{4}\pm2\mu Bmc^{2}+p^{2}c^{2}}
\end{align}
depending on the spin alignment of the particle with the magnetic field. Therefore neutrinos propagating in EM fields will not be travelling in free-particle mass eigenstates, but magnetized states.




Considering the generalization of the $2\times 2$ Pauli matrices~\cite{Ohlsson:2011zz} we obtain the covariant form 
\begin{align}       \label{cross:1a}    
        \sigma^{\alpha}=(1,+\vec{\sigma})\,,\qquad
        \bar\sigma^{\alpha}=(1,-\vec{\sigma})\,.
\end{align}
which mirrors the Weyl (chiral) representation of $\gamma^\alpha$
\begin{align}       \label{cross:1c}    
\gamma^\alpha =(off-diag I_2,I_2;\vec\sigma,-\vec\sigma)
\end{align}
This allows us to write 
        \begin{align}
        \label{cross:2}        \left(\vec{\sigma}\cdot\vec{f}_{+}\right)\left(\vec{\sigma}\cdot\vec{f}_{-}\right)=\sigma_{\alpha}\sigma_{\beta}T^{\alpha\beta}\,,\qquad
        \left(\vec{\sigma}\cdot\vec{f}_{-}\right)\left(\vec{\sigma}\cdot\vec{f}_{+}\right)=\bar\sigma_{\alpha}\bar\sigma_{\beta}T^{\alpha\beta}\,,
\end{align}
where $\sigma^{\alpha}$ and $\bar\sigma^{\alpha}$ are the Pauli 4-vectors and $T^{\alpha\beta}$ is the EM stress-energy tensor.




We note that the Lagrangian term in \req{massmom:1} is acting to the right on a left-handed chiral neutrino spinor (with the right-handed charged conjugated portion being acted upon in the Hermitian conjugate). Therefore, we can rewrite \req{massmom:1} in two-component form as
\begin{align}
    \label{massmom:1a}
    \mathcal{L}_\mathrm{eff}^\mathrm{Maj.} = 
    -\frac{1}{2}\nu_{L,2,\ell}^{T}C_{2}^{\dag}\left(M_{\ell\ell'}+\mu_{\ell\ell'}\vec{\sigma}\cdot\vec{f}_{+}\right)\nu_{L,2,\ell'}+\mathrm{h.c.}
\end{align}
where the operator $C_{2}$ is the charge conjugation matrix is written as a $2\times 2$ matrix.




Using \req{mix:1}, we write the mass-dipole matrix in \req{massmom:2} in terms of spin-flavor components as
\begin{align}
	\label{mix:2a}
    \mathcal{M}_{\ell\ell'} = 
	\begin{pmatrix}
		m_{\nu_{e}} & {\delta m}+i\delta\mu\sigma_{\alpha\beta}F^{\alpha\beta}/2\\
		{\delta m}-i\delta\mu\sigma_{\alpha\beta}F^{\alpha\beta}/2 & m_{\nu_{\mu}}
	\end{pmatrix}\equiv
    \begin{pmatrix}
        m_{\nu_{e}} & a_{+}\\
        a_{-} & m_{\nu_{\mu}}
    \end{pmatrix}\,,
\end{align}
and define the auxiliary complex variables $a_{\pm}$ and their products as
\begin{align}
    \label{mix:2b}
    a_{\pm}&={\delta m}\pm \frac{i}{2}\delta\mu(\sigma_{\alpha\beta}F^{\alpha\beta})\,,\qquad
    a_{\pm}^{\dag}={\delta m}\mp\frac{i}{2}\delta\mu(\sigma_{\alpha\beta}^{\dag}F^{\alpha\beta})\,,\\
    \label{mix:2c}
    a_{\pm}a_{\pm}^{\dag}&=\delta m^{2}\pm
    \frac{i}{2}\delta\mu\delta m\left(\sigma_{\alpha\beta}-\sigma_{\alpha\beta}^{\dag}\right)F^{\alpha\beta}+
    \frac{1}{4}\delta\mu^{2}(\sigma_{\alpha\beta}F^{\alpha\beta})(\sigma_{\alpha\beta}^{\dag}F^{\alpha\beta})\,,\\
    \label{mix:2d}
    a_{\mp}a_{\pm}^{\dag}&=\delta m^{2}\mp\frac{i}{2}\delta m\delta\mu\left(\sigma_{\alpha\beta}+\sigma_{\alpha\beta}^{\dag}\right)F^{\alpha\beta}-\frac{1}{4}\delta\mu^{2}(\sigma_{\alpha\beta}F^{\alpha\beta})(\sigma_{\alpha\beta}^{\dag}F^{\alpha\beta})\,,\\
    \label{mix:2e}
    a_{\pm}a_{\mp}&=\delta m^{2} + \frac{1}{4}\delta\mu^{2}(\sigma_{\alpha\beta}F^{\alpha\beta})^{2}\,,
\end{align}
such that \req{mix:2a} can be written more compactly. We note that the last term in $a_{\pm}a_{\pm}^{\dag}$ contain the product of $T^{\alpha\beta}$ found in \req{cross:2}. The second terms in \req{mix:2c} and \req{mix:2d} are projections of the anti-Hermitian and Hermitian parts of $\sigma_{\alpha\beta}F^{\alpha\beta}$.

\req{mix:2b} through \req{mix:2e} also reveal some interesting consequences in the presence of terms both linear in fields and quadratic in fields. The linear terms proportional to $\delta m\delta\mu\sigma F$ (with spacetime indices suppressed) involve flavor-spin rotation which is only present when off-diagonal neutrino mass terms are nonzero. The quadratic in fields terms proportional to $\delta\mu^{2}(\sigma F)^{2}$ however are induced flavor-mixing which occurs even if the neutrino mass matrix was wholly diagonal to begin with. Two limits of interest are then: (a) the weak field limit where the $\delta m\gg\delta\mu\sigma F$ such that the magnetic dipole moment only perturbs the off-diagonal mass element and (b) the strong field limit $\delta m\ll\delta\mu\sigma F$ where the magnetic dipole moment acts alone as an off-diagonal energy contribution.

The eigenvalues of \req{mix:2a} are then spin and field dependant effective masses $\widetilde{m}_{k}(E,B)$ resulting from a rotation $W$ (in contrast to $V$ in \req{diag:1}) that diagonalizes the Hermitian quantity 
\begin{align}
    \label{herm:1}
    \left(\mathcal{M}\mathcal{M}^{\dag}\right)_{\ell\ell'}&=
    \begin{pmatrix}
        |m_{\nu_{e}}|^{2}+a_{+}a_{+}^{\dag} & m_{\nu_{e}}a_{-}^{\dag}+a_{+}m_{\nu_{\mu}}^{*}\\
        a_{-}m_{\nu_{e}}^{*}+m_{\nu_{\mu}}a_{+}^{\dag} & |m_{\nu_{\mu}}|^{2}+a_{-}a_{-}^{\dag}\\
    \end{pmatrix}\equiv
    \begin{pmatrix}
        A & C\\
        C^{\dag} & B
    \end{pmatrix}\,,\\
    \label{herm:2}
    \widetilde{m}^{2}_{kk'}&= \mathrm{diag}(\widetilde{m}_{1}^{2},\widetilde{m}_{2}^{2})=W_{\ell k}^{\dag}\left(\mathcal{M}\mathcal{M}^{\dag}\right)_{\ell\ell'}W_{\ell'k'}\,.
\end{align}
This new magnetized basis is distinct from the mass basis and the flavor basis representations described in \req{basis:1}. For convenience we have defined the three unique elements $(A,B,C)$ of \req{herm:1}. The eigenvalues of \req{herm:1} are obtained from the roots of the characteristic equation
\begin{align}
    \label{secular:1}
    (A-\lambda_{k})(B-\lambda_{k})-CC^{\dag}=0\,,\qquad
    \lambda_{k}=\widetilde m_{k}^{2}\,.
\end{align}





BELOW HERE NEEDS MATH/NOTATION REVISIONS. ONLY TRUE FOR PURE B CASE: Self note: Nice rotations can be described for the pure E case, pure B case, and plane waves. General solution to rotation is illusive.

A suitable unitary $2\times2$ rotation matrix to diagonalize \req{herm:1} can be written as
\begin{alignat}{1}
	\label{mix:4} W(\theta,\phi) = 
    \left(
    \begin{array}{cc}
         \cos (\theta ) & e^{i \phi } \sin (\theta ) \\
         -e^{-i \phi } \sin (\theta ) & \cos (\theta ) \\
    \end{array}
    \right)\,,\qquad
    W_{\ell k}W^{\dag}_{\ell' k} = 1\,,
\end{alignat}
which contains one complex phase $\phi$ and one angle $\theta$. The mass-dipole diagonalization is then given by the expression
\begin{alignat}{1}
	\label{mix:5} \widetilde{m}_{kk'} = \mathrm{diag}(\widetilde{m}_{1}^{2},\widetilde{m}_{2}^{2})=W_{\ell k}^{\dag}\left(\mathcal{M}\mathcal{M}^{\dag}\right)_{\ell\ell'}W_{\ell'k'}
\end{alignat}
The assignment of the parameters $\theta$ and $\phi$ can be obtained from the

where we have defined the difference in diagonal mass elements as
\begin{align}
    \Delta m = m_{\nu_{\mu}} - m_{\nu_{e}}\,.
\end{align}
We find a suitable rotation matrix $W$ to be
\begin{multline}
    \label{w:1}
    W=\frac{|z|^{1/2}}{\left(\delta m^{2} + 4|z|^{2}\right)^{1/4}}\\
    \begin{pmatrix}
        -ie^{i\gamma/2}\frac{1}{2|z|}\left(\delta m + \sqrt{\delta m^{2} + 4|z|^{2}}\right) & -ie^{i\gamma/2}\frac{1}{2|z|}\left(\delta m - \sqrt{\delta m^{2} + 4|z|^{2}}\right)\\
        ie^{-i\gamma/2} & ie^{-i\gamma/2}        
    \end{pmatrix}
\end{multline}
which satisfies
\begin{align}
    WW^{\dag}=WW^{-1}=1\,,\qquad \mathrm{det}[W]=1\,,
\end{align}
which are the necessary conditions which for a unitary matrix.

Using \req{w:1}, the eigenvalues of the mass-dipole matrix are therefore
\begin{align}
    \label{eigenvalue:1}
    \widetilde{m}(B)_{\pm}=\frac{1}{2}\left(m_{\nu_{e}}+m_{\nu_{\mu}}\pm\sqrt{\Delta m^{2}+4{\delta m}^{2}+4\mu^{2}B^{2}}\right)\,,
\end{align}
\req{eigenvalue:1} shows that the mass splitting between the two propagating states is modified by the off-diagonal couplings in the mass matrix as well as the magnetic dipole energy $\mu B$.

If the magnetic field is set to zero $(\vec{B}=0)$, the above rotation simplifies to that of just an off-diagonal mass. The complex phase $\phi$ vanishes in a such a limit, while the rotation angle $\theta$ relaxes to the free-particle value. The inclusion of an electric field is also possible in this formation as described earlier however the mass-dipole matrix is rendered non-Hermitian and the eigenvalues more complicated in form similar to that of a complex off-diagonal mass ${\delta m}$. Mathematically both break Hermiticity in the same manner.
